\chapter{Miniproject: Adeles}\label{Adele_miniproject}

\section{Status}

This is an active miniproject.

\section{The goal}

There are several goals to this miniproject.

\begin{enumerate}
  \item Define the adeles $\A_K$ of a number field~$K$ and
    give them the structure of a $K$-algebra (status: now in mathlib thanks to
    Salvatore Mercuri);
  \item Prove that $\A_K$ is a locally compact topological ring (status:
      \href{https://github.com/smmercuri/adele-ring_locally-compact}{
      also proved by Mercuri} but not yet in mathlib);
  \item Base change: show that if $L/K$ is a finite extension of number fields then the
    natural map $L\otimes_K\A_K\to\A_L$ is an isomorphism, both algebraic and
    topological; (status: not
    formalized yet, but there is a plan -- see the project dashboard);
  \item Prove that $K \subseteq \A_K$ is a discrete subgroup and the quotient
    is compact (status: not formalized yet, but there is a plan -- see the project
    dashboard);
  \item Get this stuff into mathlib (status: (1) done, (2)--(4) not done).
\end{enumerate}

We briefly go through the basic definitions. Let $K$ be a number field.
Let $\Zhat=\projlim_{N\geq1}(\Z/N\Z)$ be the profinite completion of $\Z$,
equipped with the projective limit topology.

A cheap definition of the finite adeles $\A_K^\infty$ of $K$ is $K\otimes_{\Z}\Zhat$,
equipped with the $\Zhat$-module topology.
A cheap definition of the infinite adeles
$K_\infty$ of $K$ is $K\otimes_{\Q}\R$ with the $\R$-module topology (this is a
finite-dimensional $\R$-vector space so this is just the usual topology on $\R^n$).
A cheap definition of the adeles of $K$ is $\A_K^\infty\times K_\infty$ with
the product topology. This is a commutative topological ring.

However in the literature (and in mathlib) we see different definitions.
The finite adeles of $K$ are usually defined in the books
as the so-called restricted product $\prod'_{\mathfrak{p}}K_{\mathfrak{p}}$ over the completions
$K_{\mathfrak{p}}$ of $K$ at all maximal ideals $\mathfrak{p}\subseteq\mathcal{O}_K$ of the
integers of $K$. Here the restricted product is the subset of $\prod_{\mathfrak{p}}K_{\mathfrak{p}}$
consisting of elements which are in the integers $\mathcal{O}_{K,\mathfrak{p}}$ of
$K_{\mathfrak{p}}$ for all but finitely many $\mathfrak{p}$. This is the definition given in
mathlib. Mathlib also has the proof that they're a topological ring;
furthermore the construction of the finite adeles in mathlib works for any
Dedekind domain (this was pointed out to me by Mar\'ia In\'es
de Frutos Fern\'andez; the adeles
are an arithmetic object, but the finite adeles are an algebraic object).

Similarly the infinite adeles of a number field~$K$
are usually defined as $\prod_v K_v$,
the product running over the archimedean completions of~$K$, and this is
the mathlib definition.

The adeles of a number field $K$ are the product of the finite and infinite
adeles, and mathlib knows that they're a $K$-algebra and a topological ring.

\section{Local compactness}

As mentioned above, Salvatore Mercuri was the first to give a complete formalisation of the proof
that the adele ring is locally compact as a topological space. His work is in
\href{https://github.com/smmercuri/adele-ring_locally-compact}{his own repo} and proved the
result using the ``ad hoc'' topology on the adeles which we initially had. Since then,
adeles have been refactored to have the direct limit topology and mathlib has
\href{https://leanprover-community.github.io/mathlib4_docs/Mathlib/Topology/Algebra/RestrictedProduct.html\#RestrictedProduct.locallyCompactSpace_of_addGroup}
{\tt RestrictedProduct.locallyCompactSpace\_of\_addGroup}, the result
that a restricted product of topological additive groups $K_v$ over compact open
subgroups $A_v$ is locally compact.

What we need then is this (note that this is not true for a general Dedekind domain):
\inputleannode{NumberField.instCompactSpaceAdicCompletionIntegers}


Once we have this, the above result from mathlib gives us
\inputleannode{NumberField.AdeleRing.locallyCompactSpace}


\section{Base change}

The ``theorem'' we want is that if $L/K$ is a finite extension of number fields,
then $\A_L=L\otimes_K\A_K$. This isn't a theorem though, this is actually a \emph{definition}
(the map between the two objects) and a theorem about
the definition (that it's an isomorphism). In fact the full claim is that it is both a homeomorphism
and an $L$-algebra isomorphism. Before we can prove the theorem, we need to make the
definition.

Recall that the adeles $\A_K$ of a number field is a product $\A_K^\infty\times K_\infty$
of the finite adeles and the infinite adeles. So our ``theorem'' follows immediately from
the ``theorem''s that $\A_L^\infty=L\otimes_K\A_K^\infty$ and $L_\infty=L\otimes_KK_\infty$
(both of these equalities mean an algebraic and topological isomorphism).
We may thus treat the finite and infinite results separately.

\subsection{Base change for nonarchimedean completions.}

As pointed out above, the theory of finite adeles works fine for Dedekind domains.
So for the time being let~$A$ be a Dedekind domain. Recall that the \emph{height one spectrum}
of $A$ is the nonzero prime ideals of~$A$. Note that because we stick to the literature,
rather than to common sense, fields are Dedekind domains in mathlib, and the
height one spectrum of a field is empty. The reason I don't like allowing fields
to be Dedekind domains is that geometrically the definition of Dedekind
domain in the literature is ``smooth affine curve, or a point''. But many theorems in algebraic
geometry begin ``let $C$ be a smooth curve'', rather than ``let $C$ be a smooth curve or a point''.

Let $K$ be the field of fractions of $A$. If $v$ is in the height one spectrum of $A$,
then we can put the $v$-adic topology on $A$ and on $K$, and consider the completions
$A_v$ and $K_v$. The finite adele ring $\mathbb{A}_{A,K}^\infty$ is defined to be
the restricted product of the $K_v$ with respect to the $A_v$, as $v$ runs over
the height one spectrum of $A$. It is topologised by making $\prod_v A_v$ open with
the product topology (here $A_v$ has the $v$-adic topology).

Now let~$L/K$ be a finite separable extension, and let $B$ be the integral closure of~$A$ in~$L$.
We want to relate the finite adeles of $K$ and of $L$. We work place by place, starting by fixing
one place $w$ of $B$ and analysing the relation of $L_w$ and $B_w$ to the completions $K_v$
and $A_v$ where $v$ is the place of $A$ dividing $w$.

So let $w$ be a nonzero prime ideal of $B$. Say $w$ lies over $v$, a prime ideal of $A$.
Then we can put the $w$-adic topology on $L$ and the $v$-adic topology on~$K$. Furthermore
we can equip $K$ with an additive $v$-adic valuation, that is,
a function also called $v$ from $K$ to $\Z\cup\{\infty\}$ normalised so that if $\pi$ is a uniformiser
for $v$ then $v(\pi)=1$. Similarly we consider $w$ as a function from $L$ to $\Z\cup\{\infty\}$.
The next lemma explains how these valuations are related.

\inputleannode{IsDedekindDomain.HeightOneSpectrum.valuation_comap}


\inputleannode{IsDedekindDomain.HeightOneSpectrum.adicCompletionComapSemialgHom}

\inputleannode{IsDedekindDomain.HeightOneSpectrum.valued_adicCompletionComap}


\inputleannode{IsDedekindDomain.HeightOneSpectrum.adicCompletionComapSemialgHom.mapadicCompletionIntegers}

\inputleannode{IsDedekindDomain.HeightOneSpectrum.adicCompletionComap_isModuleTopology}


Because of the commutative diagram
\begin{center}
\begin{tikzcd}
K_v \arrow{r} & L_w  \\
K \arrow{u} \arrow{r} & L \arrow{u}
\end{tikzcd}
\end{center}

we can view $L_w$ as an $L\otimes_KK_v$-algebra.

Now instead of fixing $w$ upstairs, we fix $v$ downstairs and consider all $w$ lying
over it at once. So say $v$ is in the height one spectrum of $A$.

\inputleannode{IsDedekindDomain.HeightOneSpectrum.Extension.finite}


We write $w|v$ to denote the fact that $w$ is a prime of $B$ above $v$ of $A$.

\inputleannode{IsDedekindDomain.HeightOneSpectrum.adicCompletionComapSemialgHom'}

Because $K_v\to\prod_{w|v}L_w$ lies over $K\to L$, there's an induced $L$-algebra
map $L\otimes_KK_v\to\prod_{w|v}L_w$. We are now able to state one of the key results
in this section. The proof is probably the hardest proof
in this section to formalize.

\inputleannode{IsDedekindDomain.HeightOneSpectrum.adicCompletionComapAlgEquiv}


\inputleannode{IsDedekindDomain.HeightOneSpectrum.prodAdicCompletionComap_isModuleTopology}


\inputleannode{IsDedekindDomain.HeightOneSpectrum.adicCompletionComapContinuousAlgEquiv}


We now start thinking about what's going on at the integral level. We write $A_v$
for the integers of $K_v$ and $B_w$ for the integers of $L_w$.

\inputleannode{IsDedekindDomain.HeightOneSpectrum.adicCompletionComapAlgEquiv_integral}


A summary of what we have so far: for all finite places $v$ of $A$
we have shown that the natural map $L\otimes_KK_v\to\prod_wL_w$
is an isomorphism of $L$-algebras, and that if $L\otimes_KK_v$ has
the $K_v$-module topology and each $L_w$ has the valuation topology
then this map is also a homeomorphism. Furthermore we have shown
that there is an induced algebraic isomorphism $B\otimes_AA_v\equiv\prod_w B_w$
on the subrings of the left and right hand sides.

Recall that the finite adeles $\A_{A,K}^\infty$ is defined in mathlib to be
the restricted product of the $K_v$ with respect to the $A_v$, equipped with a certain
restricted product topology (which is not the subspace topology of the product
topology, indeed $\prod_v A_v$ is open in this topology). We have seen in
definition~\ref{IsDedekindDomain.HeightOneSpectrum.adicCompletionComapSemialgHom} that
there's a map $K_v\to L_w$ if $w|v$, extending $K\to L$, and we have seen in
theorem~\ref{IsDedekindDomain.HeightOneSpectrum.adicCompletionComapSemialgHom.mapadicCompletionIntegers}
that this sends $A_v$ to $B_w$. We conclude

\inputleannode{IsDedekindDomain.FiniteAdeleRing.mapSemialgHom}

Hence there's a natural $L$-algebra homomorphism $L\otimes_K\A_{A,K}^\infty\to\A_{B,L}^\infty$.

Our next goal in this section is the following two results. First the algebraic claim:

\inputleannode{IsDedekindDomain.FiniteAdeleRing.baseChangeAlgEquiv}

Now $L\otimes_K\A_{A,K}^\infty$ is an $\A_{A,K}^\infty$-module and hence can be given
the $\A_{A,K}^\infty$-module topology. We also claim

\inputleannode{IsDedekindDomain.FiniteAdeleRing.baseChangeContinuousAlgEquiv}

Informally, the proofs are simple: componentwise we know
that $L\otimes_KK_v$ is isomorphic both algebraically and
topologically to $\prod_{w|v}L_w$, and that this isomorphism
sends the open set $B\otimes_AA_v$ homeomorphically onto
$\prod_{w|v}B_w$, so now it's ``just a case of putting everything
together''. Formally, we really need to spell this out, as there is a lot
going on. We do this in the
next subsection.

\subsection{Base change for nonarchimedean completions.}

As usual we are in the AKLB set-up, so in particular $K$ is the field
of fractions of the Dedekind domain $A$, $L/K$ is a finite
separable extension, and $B$ is the integral closure of $A$
in $L$. The goal in this subsection is to spell out the following argument: Assume that
$L\otimes_KK_v\cong\prod_{w|v}L_w$
algebraically and topologically for all $v$, with $B\otimes_AA_v$ identified with $\prod_{w|v}B_w$.
Then $L\otimes_K\A_K^\infty\cong\A_L^\infty$, algebraically and topologically. Here
the tensor products $L\otimes_K R$ (for $R$ a $K$-algebra with a topology) are all being
given the $R$-module topology, which if we choose a basis for $L/K$ is just the product
topology.

We start with the following observation. If $M$ is a $K$-module then there's a canonical
map $B\otimes_AM\to L\otimes_KM$ sending $b\otimes m$ to $b\otimes m$ (this follows from the
universal property of the tensor product). Our first goal
is to show that this map is an isomorphism. Let us establish some lemmas first.

\begin{lemma}
  \label{IsDedekindDomain.dvd_norm}
  If $0\not=b\in B$ then there exists $0\not=a\in A$ such that $b$ divides
  the image of $a$ in $B$.
\end{lemma}
\begin{remark} Is this already in mathlib?
\end{remark}
\begin{proof} Let $a=N_{L/K}(b)$, the norm. This is known to take nonzero elements of $L$
to nonzero elements of $K$ (because the norm is the determinant of an invertible linear map)
and integral elements to integral elements. Furthermore $a/b\in L$ is the the product of the
conjugates of $b$ in some normal closure of $L$, and hence it is integral, and thus in $B$.
\end{proof}

\begin{corollary}
  \label{IsDedekindDomain.AKLB.surjective_tensorProduct_map}
  \uses{IsDedekindDomain.dvd_norm}
  The $A$-bilinear map $B\times K\to L$ sending $(b,k)$ to $bk$ is surjective.
\end{corollary}
\begin{proof} Given $\lambda\in L$ write it as $n/d$ with $0\not=d\in B$. Choose $0\not=a\in A$
  and $b\in B$ with $db=a$ and then note $\lambda=nb/a=nb\times a^{-1}$.
\end{proof}

\begin{corollary}
  \label{IsDedekindDomain.FiniteAdeleRing.tensorProduct_algEquiv}
  \uses{IsDedekindDomain.dvd_norm}
  The natural map $B\otimes_AK\to L$ is a $B$-algebra isomorphism.
\end{corollary}
\begin{proof}

We write down an inverse. Regard $B\otimes_AK$ as a $B$-algebra via the action on the left.
Note that at this point it's not even clear that $B\otimes_AK$ is a field. We have the
structure map $B\to B\otimes_AK$ sending $b$ to $b\otimes1$, which is $B$-linear. I claim
that every nonzero element of $B$ gets sent to an invertible element of $B\otimes_AK$.
Indeed, if $b\not=0$ and (using the previous lemma) we choose $0\not=a\in A$ such that
$bb'=a$, then $(b\otimes1)(b'\otimes\frac1a)=1$. Thus by the universal property of
localisation, the $B$-linear map $B\to B\otimes_AK$ extends to a ring homomorphism from the field
of fractions of $B$ to $B\otimes_AK$, which we claim is our desired inverse.
Checking that both composites are the identity should be straightforward. Starting
with $B\otimes_AK$ we only have to check on elements of the form $b\otimes k$;
starting with $L$ we only have to check on elements of $B$. Hopefully both are
straightforward.
\end{proof}

\inputleannode{IsDedekindDomain.AKLB.tensorProduct_module_algEquiv}


We now need to explain how tensor products sometimes commute with restricted products.
Something we will need along the way is

\begin{theorem} $B$ is a finitely-presented $A$-module.
  \label{IsDedekindDomain.AKLB.finitePresentation}
\end{theorem}
\begin{proof} $A$ is Noetherian as it is a Dedekind domain, so it suffices to prove that $B$ is
  finitely-generated as an $A$-module. But this is in mathlib already (a proof is
  around line 101 of {\tt BaseChange.lean} in FLT at the time of writing).
\end{proof}

The reason we care about this is the following.

\begin{theorem}
  \label{pi_tensorProduct_of_finitePresentation}
  If $R$ is a commutative ring, if $M$ is a finitely presented $R$-module
  and if $N_i$ are a collection of $R$-modules, then the canonical map
  $M\otimes_R\prod_i N_i\to\prod_i(M\otimes_R N_i)$ is an isomorphism.
\end{theorem}
\begin{proof} If $M$ is finite and free then Maddy Crim has already formalized this
  in FLT. For the general case present $M$ as $R^a\to R^b\to M\to 0$ and use that tensor
  products and arbitrary products preserve surjections.
\end{proof}

\begin{corollary}
  \label{IsDedekindDomain.pi_tensorProduct}
  \uses{IsDedekindDomain.AKLB.finitePresentation,pi_tensorProduct_of_finitePresentation}
  If $S$ is a finite set of nonzero primes of $A$ then the natural map
  $B\otimes((\prod_{v\in S}K_v)\times(\prod_{v\notin S}A_v))\to
  (\prod_{v\in S}(B\otimes_AK_v))\times(\prod_{v\notin S}(B\otimes_AA_v))$
  is an isomorphism.
\end{corollary}
\begin{proof} Follows from the previous two theorems.
\end{proof}

Recall that $\A_K^\infty$ is the finite adeles of $K$,
defined as the restricted product of the $K_v$ with respect to the $A_v$,
where $v$ runs through the nonzero primes of $A$. Let $R$ denote the restricted
product of the $B\otimes_A K_v$ with respect to the $B\otimes_A A_v$.

\begin{corollary}
  \label{IsDedekindDomain.FiniteAdeleRing.IntegraltensorProductAlgEquiv_aux1}
  \uses{IsDedekindDomain.pi_tensorProduct}
  The natural map $B\otimes_A\A_K^\infty\to R$ is a $B$-algebra isomorphism.
\end{corollary}
\begin{proof} This follows from the previous corollary and the fact that
  tensor products commute with filtered colimits.
\end{proof}

Recall from earlier in this section that if $v$ is a finite place of $A$ then the natural map from
$B\otimes_A K_v$ to $L\otimes_KK_v$ is an isomorphism, and recall from the previous section
that the natural map from $L\otimes_KK_v$ to $\prod_{w|v}L_w$ was also an isomorphism.
This isomorphism sends $B\otimes_A A_v$ to $\prod_{w|v}B_w$ (I thank Matthew Jasper for
pointing out to me that this statement was true at all primes, not just at unramified primes).
Finally, the set of $w$ of $B$ dividing a fixed place $v$ of $A$ is finite.
Let's now formalize the abstract statement which we need. Rather than following the
notation for restricted product in the literature and writing $\A_K^\infty=\prod'_vK_v$
with the $\mathcal{O}_v$ implicit, we will write $\prod'_v(K_v,\mathcal{O}_v)$ in the below.

\begin{definition}
  \label{RestrictedProduct.relabelIso}
  Let $V$ and $W$ be index sets, and let $f:W\to V$ be a map with finite fibres.
  Let $X_v$ be sets, with subsets $C_v$, let $Y_w$ be sets with subsets $D_w$,
  and say for all $v\in V$ we're given a bijection $X_v\to\prod_{w|f(w)=v}Y_w$,
  identifying $C_v$ with $\prod_{w:f(w)=v}D_w$. Then there's an induced bijection between
  the restricted products $\prod'_v(X_v,C_v)$ and $\prod'_w(Y_w,D_w)$.
\end{definition}

\begin{corollary}
  \label{IsDedekindDomain.FiniteAdeleRing.IntegraltensorProductAlgEquiv_aux2}
  \uses{RestrictedProduct.relabelIso}
  The ring $R$ introduced above (the restricted
  product of the $B\otimes_A K_v$ with respect to the $B\otimes_A A_v$)
  is isomorphic to $\mathbb{A}_L$.
\end{corollary}
\begin{proof} Let $V$ be the finite places of $K$ and $W$ the finite places of $L$,
  let $X_v$ be $B\otimes_A K_v$, let $C_v$ be $B\otimes_A A_v$, let $Y_w$ be $L_w$,
  let $D_w$ be $B_w$ and the result follows from the previous definition, given
  theorem~\ref{IsDedekindDomain.HeightOneSpectrum.adicCompletionComapAlgEquiv_integral}.
\end{proof}
From this, we can deduce the theorem we claimed earlier:

\begin{theorem}
  \label{IsDedekindDomain.FiniteAdeleRing.baseChangeIntegralAlgEquiv}
  \uses{RestrictedProduct.relabelIso,
  IsDedekindDomain.FiniteAdeleRing.IntegraltensorProductAlgEquiv_aux1,
  IsDedekindDomain.FiniteAdeleRing.IntegraltensorProductAlgEquiv_aux2}
  The natural map $B\otimes_A\A_K^\infty\to\A_L^\infty$ is a $B$-algebra
  isomorphism.
\end{theorem}
\begin{proof}
  This map factors through the auxiliary ring~$R$ so the result follows
  from the previous two constructions.
\end{proof}

Because this map factors through the isomorphism $B\otimes_A\A_K^\infty\to L\otimes_K\A_K^\infty$
we can finally deduce that the natural map $L\otimes_K\A_K^\infty\to\A_L^\infty$ is an algebraic
  isomorphism.


We still need to talk about topologies though, so let's finish by doing this. Let's start with some trivialities, expressed as definitions rather than theorems
because they're constructions.

\begin{definition} If $X_v$ and $Y_v$ are families of topological spaces indexed by $v\in V$,
  if $f_v:X_v\to Y_v$ is a continuous map sending the subset $C_v\subseteq X_v$ into
  $D_v\subseteq Y_v$ then there's an induced continuous map $\prod'_v(X_v,C_v)\to\prod'_v(Y_v,D_v)$.
\end{definition}

\begin{definition} If all the $f_v$ are homeomorphisms identifying $C_v$ with $D_v$ then
  the induced map on restricted products is also a homeomorphism (proof: apply the previous
  construction to $f_v$ and their inverses)
\end{definition}

We now allow a slight change of index set. Unfortunately I don't think that we can deduce
the results just stated above from this one, in Lean, because the product of $Y_w$ over a
set of size 1 is not definitionally equal to $Y_w$.

Recall definition~\ref{RestrictedProduct.relabelIso}, giving us a bijection between two restricted
products.

\begin{theorem} In the same setup as definition~\ref{RestrictedProduct.relabelIso}
  ($V,W$ index sets, $f:W\to V$,
  $C_v\subseteq X_v$ and $D_w\subseteq Y_w$, bijections $b_v:X_v\to\prod_{w:f(w)=v}Y_w$
  identifying $C_v$ with $\prod_{w:f(w)=v}D_w$), if all the $X_v$ and $Y_w$ are furthermore
  topological spaces, all the $C_v$ and $D_w$ are open, and all the $b_v$ are homeomorphisms,
  then the induced
  map $\prod'_v(X_v,C_v)\to\prod'_w(Y_w,D_w)$ is also a homeomorphism.
\end{theorem}
\begin{proof} I have only thought about the cofinite filter case, where this
  should follow easily from the definition of the topology.
\end{proof}

\begin{corollary} $\mathbb{\A_L^\infty}$ is homeomorphic to $\prod_v(B\otimes_AK_v,B\otimes_AA_v)$.
\end{corollary}
\begin{proof} Follows from the previous theorem with $X_v=B\otimes_AK_v$ $D_w=L_w$ etc.
\end{proof}

Recall that if $$R$$ is a commutative ring, and two $$R$$-modules both have the $$R$$-module
topology, then any $$R$$-linear morphism between them is automatically continuous. We know
that $\A_L^\infty$ is $\A_K^\infty$-linearly isomorphic to $L\otimes_K\A_K^\infty$ and our claim is that
if $L\otimes_K\A_K^\infty$ is given the $\A_K^\infty$-module topology then this isomorphism is also a
homeomorphism; to prove this, we thus just need to check that $\A_L^\infty$ has the $\A_K^\infty$-module
topology. Equivalently, by the previous result, we need to check that
the restricted product topology on the $\A_K^\infty$-algebra $\prod'_v(B\otimes_AK_v,B\otimes_AA_v)$
is the $\A_K^\infty$-module topology.

We now need to make restricted products of modules into modules over restricted product of rings.
The API, which should be straightforward so we don't give details here, is: if $R_v$ are rings
with subrings $S_v$ and if $M_v$ are $R_v$-modules with $S_v$-stable submodules $N_v$, then
$\prod'_v(M_v,N_v)$ is naturally a module over $\prod'_v(R_v,S_v)$, and that $R_v$-morphisms
$M_v\to M_v'$ sending $N_v$ to $N_v'$ induce $\prod'_v(R_s,S_v)$-linear maps
$\prod'_v(M_v,N_v)\to\prod'_v(M'_v,N'_v)$. From this one deduces that isomorphisms on the factors
induce isomorphisms on the restricted products.


Now $A_v$ is a PID, so $B\otimes_AA_v$ is free (as it is finitely-generated and torsion-free).
This means that there is an isomorphism $B\otimes_AA_v\cong(A_v)^n$, which extends to an isomorphism
$B\otimes_AK_v\cong K_v^n$. These isomorphisms are also homeomorphisms. If we fix such isomorphisms
for all $v$ then we get an induced $\A_K^\infty$-module isomorphism + homeomorphism
$\prod'_v(B\otimes_AK_v,B\otimes_AA_v)=\prod'_v(K_n^n,A_v^n)$. So it suffices to prove
that the $\prod'_v(K_v,A_v)$-module $\prod'_v(K_v^n,A_v^n)$ has the $\prod'_v(K_v,A_v)$-module
topology. This follows from the fact that the product topology on two modules with the
module topology is the module topology (a fact in mathlib) and the following result.

\begin{lemma}
  If $X_v$ and $Y_v$ are topological spaces with open subspaces $C_v$ and $D_v$, then
  the obvious bijection $\prod'_v(X_v \times Y_v,C_v\times D_v) \cong
  \left(\prod'_v(X_v,C_v)\right)\times\left(\prod'_v(Y_v,D_v)\right)$ is a homeomorphism,
  where the restricted products have the restricted product topology and the binary
  product has the product topology.
\end{lemma}
\begin{proof} This should hopefully be straightforward using {\tt RestrictedProduct.continuous\_dom\_prod}
\end{proof}

As a corollary one can prove by induction on $n$ that the restricted product of $n$th powers
is homeomorphic to the $n$th power of the restricted product and this is the result
we require.

\subsection{Base change for infinite adeles}

Recall that if $K$ is a number field then the infinite adeles of $K$ are defined
to be the product $\prod_{v\mid\infty} K_v$ of all the completions of $K$ at the
infinite places.

The result we need here is that if $L/K$ is a finite extension of number fields,
then the map $K\to L$ extends to a continuous $K$-algebra map $K_\infty\to L_\infty$,
and thus to a continuous $L$-algebra isomorphism $L\otimes_KK_\infty\to L_\infty$.
Perhaps a cheap proof would be to deduce it from the fact that $K_\infty=K\otimes_{\Q}\R$.

The overall strategy is to first establish, for each infinite place $v$ of $K$, homeomorphisms
between for the completion $K_v$ and the product $\prod_{w\mid v}L_w$ of completions of $L$ at
all infinite places $w$ of $L$ lying above $v$.
We then use these homeomorphisms to construct base change for the infinite adele ring.

\subsubsection{Weak approximation at infinite places}

First, we require a preliminary result that $K$ is dense inside any product of
completions $\prod_{v\in S} K_v$ of $K$ at infinite places.

\inputleannode{NumberField.InfinitePlace.Completion.denseRange_algebraMap_subtype_pi}


\subsubsection{Dimensionality of $\prod_{w\mid v}L_w$ as a $K_v$-vector space}

This subsection contains a result that the $K_v$-dimension of
$L \otimes_K K_v$ is equal to the $K_v$-dimension of $\prod_{w\mid v}L_w$.

\inputleannode{NumberField.InfinitePlace.Completion.finrank_pi_eq_finrank_tensorProduct}

\subsubsection{Base change at infinite places}

\inputleannode{NumberField.InfinitePlace.Completion.piExtension}

The map in~\ref{NumberField.InfinitePlace.Completion.piExtension} can be lifted to an
$L$-algebra homomorphism defined on $L\otimes_K K_v$.

\inputleannode{NumberField.InfinitePlace.Completion.baseChange}

\inputleannode{NumberField.InfinitePlace.Completion.baseChange_surjective}


\inputleannode{NumberField.InfinitePlace.Completion.baseChange_injective}


We have established that the map of
Definition~\ref{NumberField.InfinitePlace.Completion.baseChange} gives an $L$-algebra isomorphism
between $L\otimes_K K_v$ and $\prod_{w\mid v}L_w$.
The left-hand side is given the $K_v$-module topology, while we show that the right-hand side also
has the $K_v$-module topology.
\inputleannode{NumberField.InfinitePlace.Completion.instIsModuleTopologyValEqComapAlgebraMap_fLT}


\inputleannode{NumberField.InfinitePlace.Completion.baseChangeEquiv}


\inputleannode{NumberField.InfinitePlace.Completion.piEquiv}


\subsubsection{Base change for the infinite adele ring}

First, we induce a $K_{\infty}$-algebra on $L_{\infty}$ from the action of each $K_v$ on
$\prod_{w\mid v}L_w$.
Specifically, this means that for $x \in K_{\infty}$ and $y \in L_{\infty}$, we have
$(x \cdot y)_w = x_{v_w} \cdot y_w$, where $v_w$ is the restriction of $w$ to $K$.
We show that $L_{\infty}$ has the $K_{\infty}$-module topology.

\inputleannode{NumberField.InfiniteAdeleRing.piEquiv}


\inputleannode{NumberField.InfiniteAdeleRing.instIsModuleTopology_fLT}


\inputleannode{NumberField.InfiniteAdeleRing.baseChangeEquivAux}


It remains to show that the map in~\ref{NumberField.InfiniteAdeleRing.baseChangeEquivAux} is a
homeomorphism.
Since both sides have the $K_{\infty}$-module topology, and since the $L$-algebra isomorphism
of~\ref{NumberField.InfiniteAdeleRing.baseChangeEquivAux} can equivalently be viewed as a
$K_{\infty}$-linear isomorphism, it is also a homeomorphism.

\inputleannode{NumberField.InfiniteAdeleRing.baseChangeEquiv}



\subsection{Base change for adeles}

From the previous results we deduce immediately that if $L/K$ is a finite extension
of number fields then there's a natural (topological and algebraic) isomorphism
$L\otimes_K\A_K\to \A_L$.

\inputleannode{NumberField.AdeleRing.baseChangeEquiv}


Something else we shall need:

\inputleannode{NumberField.AdeleRing.baseChange_moduleTopology}


\section{Discreteness and compactness}

We need that if $K$ is a number field then
$K\subseteq\mathbb{A}_K$ is discrete, and the quotient (with the
quotient topology) is compact. Here is a proposed proof.

\inputleannode{Rat.AdeleRing.zero_discrete}


\inputleannode{NumberField.AdeleRing.zero_discrete}


\inputleannode{NumberField.AdeleRing.discrete}


For compactness we follow the same approach.

\inputleannode{Rat.AdeleRing.cocompact}


\inputleannode{NumberField.AdeleRing.cocompact}

