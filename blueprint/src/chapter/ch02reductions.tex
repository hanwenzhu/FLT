\chapter{First reductions of the problem}\label{ch_reductions}

\section{Goal}

The goal of this chapter is to reduce FLT to a deep theorem of Mazur and a deep theorem of Wiles
about a Galois representation.

\section{Overview}
The proof of Fermat's Last Theorem is by contradiction. We assume that we have a counterexample
$a^n+b^n=c^n$, and manipulate it until it satisfies the axioms of a ``Frey package'', a basic
concept which we will explain below. From the
Frey package we build a Frey curve -- an elliptic curve defined over the rationals. We then look at
a certain representation of a Galois group coming from this elliptic curve, and finally using two
very deep and independent theorems (one due to Mazur, the other due to Wiles) we show that this
representation is both reducible and irreducible, the contradiction we seek.

\section{Reduction to \texorpdfstring{$n\geq5$}{ngeq5} and prime}

\inputleannode{FermatLastTheorem.of_odd_primes}


Euler proved Fermat's Last Theorem for $p=3$;

\inputleannode{fermatLastTheoremThree}


\inputleannode{FermatLastTheorem.of_p_ge_5}


\section{Frey packages}

For convenience we make the following definition.

\inputleannode{FreyPackage}

Our next reduction is as follows:

\inputleannode{FreyPackage.of_not_FermatLastTheorem_p_ge_5}


\section{Galois representations and elliptic curves}\label{twopointfour}

To continue, we need some of the theory of elliptic curves over $\Q$. So let $f(X)$ denote any
monic cubic polynomial with rational coefficients and whose three complex roots are distinct,
and let us consider the equation $E:Y^2=f(X)$, which defines a curve in the $(X,Y)$ plane.
This curve (or strictly speaking its projectivisation) is a so-called elliptic curve
(or an elliptic curve over $\Q$ if we want to keep track of the field where the coefficients
of $f(X)$ lie).

If $E:Y^2=f(X)$ is an elliptic curve over $\Q$, and if $K$ is any characteristic zero field (and
hence a $\Q$-algebra), then we write $E(K)$ for the set of solutions to $y^2=f(x)$ with $x,y\in K$,
together with an additional ``point at infinity'' corresponding morally to $x=y=\infty$.
It is an extraordinary fact,
and not at all obvious, that $E(K)$ naturally has the structure of an additive abelian group,
with the point at infinity being the zero element (the identity). Fortunately this fact is
already in {\tt mathlib}. This additive group structure has
the property that three distinct points $P,Q,R\in K^2$ which are in $E(K)$ will sum to zero
if and only if they are collinear.

The group structure behaves well under change of field: if $E$ is an elliptic curve over $\Q$
and if $K\to L$ is a homomorphism of
characteristic zero fields then the induced map $E(K)\to E(L)$ is a group homomorphism.
Thus if $f:K\to L$ is an isomorphism of characteristic zero fields, the induced map $E(K)\to E(L)$
is an isomorphism of groups, with the inverse isomorphism being the map $E(L)\to E(K)$ induced
by $f^{-1}$. This construction thus gives us an action of the multiplicative group $\Aut(K)$
of automorphisms of the field $K$ on the additive abelian group $E(K)$, and hence also
on the $n$-torsion of this group for any positive integer~$n$.
In particular, if $\Qbar$ denotes an algebraic closure of the
rationals (for example, the algebraic numbers in $\bbC$) and if $\GQ$ denotes the group of field
isomorphisms $\Qbar\to\Qbar$, then for any elliptic curve $E$ over $\Q$ we have an action
of $\GQ$ on the additive abelian group $E(\Qbar)$, and hence on its $n$-torsion subgroup
$E(\Qbar)[n]$.

If furthermore $n=p$ is prime, then $E(\Qbar)[p]$ is naturally a vector space over the
field $\Z/p\Z$, and thus it inherits the structure of a mod $p$ representation of $\GQ$.
This is the \emph{mod $p$ Galois representation} attached to the elliptic curve $E$.
It is well-known to be 2-dimensional. We call this representation $\rho_{E,p}$.

In the next section we apply this theory to an elliptic curve coming from a counterexample to
Fermat's Last theorem.

\section{The Frey curve}

Recall that a \emph{Frey package} $(a,b,c,p)$ is simply a prime $p\geq5$ and nonzero
pairwise-coprime integers $a,b,c$ satisfying $a^p+b^p=c^p$ and satisfying the congruences
$a\equiv3\pmod4$ and $b\equiv0\pmod2$. We have shown above that if Fermat's Last Theorem is false,
then a Frey package exists.

\inputleannode{FreyCurve}

Note that the roots of the cubic $X(X-a^p)(X+b^p)$ are distinct because $a,b,c$ are nonzero and
$a^p+b^p=c^p$.

Given a Frey package $(a,b,c,p)$ with corresponding Frey curve $E$, the mod $p$ Galois
representation $\rho_{E,p}$ associated to this package is the 2-dimensional representation of
$\GQ$ on $E(\Qbar)[p]$ described above. Frey's observation is that this mod $p$ Galois
representation has some very surprising properties. We will make this remark more explicit
in the next chapter. Here we shall show how these properties can be used to finish the job.

\section{Reduction to two big theorems.}

Recall that a representation of a group $G$ on a vector space $W$ is said to be \emph{irreducible}
if there are precisely two $G$-stable subspaces of $W$, namely $0$ and $W$.
The representation is said to be \emph{reducible} otherwise.

Now say $(a,b,c,p)$ is a Frey package.
Consider the mod $p$ representation of $\GQ$ coming from the $p$-torsion in the Frey
curve $Y^2=X(x-a^p)(X+b^p)$ associated to the package. Let's call this representation $\rho$,
and we say that $\rho$ is the mod $p$ representation associated to the Frey package $(a,b,c,p)$.
Is it irreducible or not?

\inputleannode{Mazur_Frey}


Note that in the first (pre-2029) phase of the FLT project, we will not be working on
a formalization of this result, as it was known in the 1980s. We will however be thinking
a lot about the next result, which says the exact opposite.

\inputleannode{Wiles_Frey}


\inputleannode{FreyPackage.false}


We deduce

\inputleannode{FLT}


Because we are (for now at least) assuming Mazur's theorem, we now need to turn our attention
to a proof of theorem~\ref{Wiles_Frey}. We start on this proof in Chapter~\ref{ch_freyreduction}.
