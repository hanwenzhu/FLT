\chapter{Reducibility of {\it p}-torsion of the Frey curve}\label{ch_freyreduction}

\section{Overview}

In chapter~\ref{ch_reductions} we reduced FLT, modulo a hard theorem from the 1970s,
to Theorem~\ref{Wiles_Frey}, the assertion that $p$-torsion in the Frey
curve is reducible. In this chapter we deduce this assertion from three more complex claims
about ``hardly ramified'' Galois representations. It is relatively straightforward
to reduce one of these three claims to a result of Fontaine proved in the 1980s in his
paper on the nonexistence of nontrivial abelian schemes over $\Z$. The other two
claims lie deeper, and their proofs use techniques initially developed by Wiles in
the 1990s.

\section{Hardly ramified representations}

Let $(a,b,c,p)$ be a Frey package (so in particular $p\geq5$ is prime and $a^p+b^p=c^p$),
let $E$ be the corresponding Frey curve over $\Q$, and let $\rho:\GQ\to\Aut(E(\Qbar)[p])$
be the 2-dimensional Galois representation on the $p$-torsion of~$E$. Recall that our goal
is to prove that $\rho$ is reducible.

What we need to leverage is the fact that $\rho$ has very little ramification. To give
a toy example before we start: if $K$ is a number field (i.e., a finite extension of $\Q$)
and if the extension $K/\Q$ is unramified at all primes, then an old theorem of
Minkowski tells us that $K=\Q$. We want to prove a theorem in a similar vein, namely
that if a 2-dimensional mod $p$ Galois representation is ``hardly ramified'', then it is reducible.
Below, we give a precise
definition of what it means for a continuous 2-dimensional representation $\GQ\to\GL_2(R)$
to be hardly ramified. Before we do that, we need to say precisely which topological rings~$R$
we will allow. We say that a topological ring is emph{pro-Artinian} if it is a projective
limit of Artin local rings each equipped with the discrete topology, and if it has the
projective limit topology. We are only concerned with local pro-Artinian rings with finite
residue field; such things can be checked to be the same thing as topological
local rings with finite residue field whose underlying topological space is profinite,
and such that additive translates of open ideals form a basis for the topology.
Let us call such rings ``coefficient rings'' for now.

\begin{remark} We make some remarks to orient the reader.
  \begin{itemize}
    \item Any complete local Noetherian ring with finite residue field is a coefficient ring,
      if the ring is equipped with the $\m$-adic topology where $\m$ is the maximal ideal.
      In this case, all powers of $\m$ are open.
    \item In particular finite fields, and integer rings of finite extensions
      of $\Q_p$, are coefficient rings.
    \item If $R$ is a coefficient ring then $R$ is isomorphic to the projective limit
      of the finite rings $R/I$ as $I$ runs over the open ideals of~$R$.
    \item A non-Noetherian example of a coefficient ring is the projective limit over $n$ of
the rings $\Z/p\Z[\varepsilon_1,\ldots,\varepsilon_n]/(\forall i,j,\varepsilon_i\varepsilon_j=0)$;
these rings are convenient to include as coefficient rings for technical reasons; they make
representability theorems easier.
    \item The category of coefficient rings is equivalent to the pro-category of the
      category of finite local rings.
    \item A coefficient ring is pseudocompact in the sense of Grothendieck. A pseudocompact
      local ring is however a more general concept as such a thing may have an infinite
      residue field and would thus not be profinite.
    \item If $R$ is a coefficient ring with residue field of characteristic $\ell$,
  then there is a unique continuous map $\Z_\ell\to R$. Indeed, it suffices to prove that there
  is a unique continuous map $\Z_\ell\to R/I$ for each
  open ideal~$I$, but $R/I$ is a finite local ring with residue field of characteristic $\ell$.
  $R/I$ is hence Artinian, so some power of the maximal ideal is zero by Nakayama. This means
  that $\ell^N=0$ for some sufficiently large $N$, and hence $R/I$ is a $\Z/\ell^N\Z$-algebra
  and thus admits admits a unique map from $\Z_\ell$.
    \item It will be more convenient to fix once and for all the integer $\mathcal{O}$ in a
  finite extension of $\Q_\ell$ and consider ``coefficient $\mathcal{O}$-algebras'', namely
  coefficient rings~$R$ equipped with a continuous map $\mathcal{O}\to R$ which is a local
  homomorphism inducing an isomorphism on residue fields.
  \end{itemize}
\end{remark}

Because a coefficient ring~$R$ with residue field of characteristic $\ell$ is naturally
a $\Z_\ell$-algebra, we can talk about the $\ell$-adic cyclotomic character $\GQ\to R^\times$.
We are now ready to define hardly ramified representations.

\inputleannode{hardly_ramified}

A well-known result, which basically goes back to Frey, is the following:

\inputleannode{Frey_curve_hardly_ramified}



Note that irreducibility and absolute irreducibility for hardly ramified mod $\ell$ representations
are the same, because our assumptions that $\ell\geq3$
and that the determinant is cyclotomic imply that the image of complex conjugation
has distinct eigenvalues defined over the ground field.

The key theorem about hardly ramified representations is the following.

\inputleannode{hardly_ramified_reducible}

Note that this (deep) claim is a consequence of Serre's conjecture~\cite{serreconj},
now a theorem of Khare and Wintenberger~\cite{kwII}, and indeed we shall use
methods introduced by Khare and Wintenberger to prove this special case of
Serre's conjecture. Given this result, we can deduce Theorem~\ref{Wiles_Frey}
(which we restate here) easily:

\inputleannode{Wiles_Frey}


Our job of reducing FLT to theorems of the 1980s is hence reduced to proving
Theorem~\ref{hardly_ramified_reducible}.

\subsection{Hardly ramified mod $p$ representations are reducible}

In this section we will state three theorems, from which Theorem~\ref{hardly_ramified_reducible}
easily follows.

Firstly, we claim that
an irreducible hardly ramified mod $\ell$ representation lifts to an $\ell$-adic representation.

\inputleannode{hardly_ramified_lifts}


Next we claim that a hardly ramified $\ell$-adic representation ``spreads out'' to a compatible
family of hardly ramified $q$-adic representations for all odd primes $q$ (note that we have
not made a definition of a hardly ramified 2-adic representation).

\inputleannode{hardly_ramified_spreads_out}


In particular, we can ``move'' from an irreducible hardly ramified mod $\ell$ representation
to a hardly ramified 3-adic representation, and hence to a hardly ramified mod 3 representation.

However, we can essentially completely classify the hardly ramified mod 3 Galois representations:

\inputleannode{hardly_ramified_mod3_reducible}


And we can use this to essentially completely classify the hardly ramified 3-adic Galois
representations:

\inputleannode{hardly_ramified_3adic_reducible}


Theorem~\ref{hardly_ramified_reducible} (if $\ell\geq 3$ is a prime and
$\overline{\rho}:\GQ\to\GL_2(\Z/\ell\Z)$ is hardly ramified,
then $\overline{\rho}$ is reducible) is an easy consequence of these theorems,
as we now show.



What remains then (modulo several results which were known in the 1980s),
is to prove the three theorems~\ref{hardly_ramified_lifts},
\ref{hardly_ramified_spreads_out} and~\ref{hardly_ramified_3adic_reducible}.
By far the easiest is theorem~\ref{hardly_ramified_3adic_reducible}; this follows
from old estimates of Fontaine (ultimately relying on bounds for root discriminants due to
Odlyzko and Poitou), originally developed to prove that there was no
nontrivial abelian scheme over $\Z.$ The other two theorems are deeper, and both use
modern variants of Wiles' $R=T$ machinery.

We have not yet written any more LaTeX on how to proceed further; the rest of
this blueprint should be considered as more unfocussed thoughts.
