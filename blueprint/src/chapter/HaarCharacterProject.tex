\chapter{Miniproject: Haar Characters}\label{Haar_char_project}

\section{The goal}

The goal of this miniproject is to develop the theory (i.e., the basic API) of Haar characters.
``Haar character'' is a name I've made up to describe a certain character of the units of a locally
compact topological ring. The main result we need here is that if $B$ is a finite-dimensional
algebra over a number field~$K$, then $B^\times$ is in the kernel of the Haar character
of $B\otimes_K\A_K$, where $\A_K$ is the ring of adeles of~$K$. Most if not all of this
should probably be in mathlib.

KMB would like to heartily thank S\'ebastien Gou\"ezel for the help he gave during the preparation
of this material.

\section{Initial definitions}

\subsection{Scaling Haar measure on a group}

Let $A$ be a locally compact topological additive abelian group. There's then a regular additive
Haar measure $\mu$ on $A$, unique up to a positive scalar factor. If $\phi:(A,+)\cong(A,+)$ is a
homeomorphism and an additive automorphism of $A$, then we can push forward $\mu$
along $\phi$ to get a second measure $\phi_*\mu$ on $A$, with the property that
$(\phi_*\mu)(X)=\mu(\phi^{-1}X)$ for any Borel subset $X$ of $A$.

Now $\phi_*\mu$ is a translation-invariant and regular measure,
and hence also a Haar measure on $A.$ It must thus differ from
$\mu$ by a positive scalar factor, which we call $d_A(\phi)$.
There is a choice of normalization here between $d_A(\phi)$
and $d_A(\phi)^{-1}$, so let us be more precise.

\inputleannode{MeasureTheory.addEquivAddHaarChar}

To give an example, if $\phi$ is multiplication by $2$ on the real numbers,
if $X=[0,1]$, and if $\mu$ is Lebesgue measure on the Borel subsets of $\R$,
we have that $\phi_*\mu(X)=\mu(\phi^{-1}(X))=\mu([0,1/2])=1/2$,
so $1=d_A(\phi)/2$ meaning that $d_A(\phi)=2$. Similarly if $\phi$ is multiplication
by $-2$ and $X=[0,1]$ then $\phi^{-1}(X)=[-1/2,0]$ which again has measure $1/2$,
so $d_A(\phi)$ is 2 again.

Strictly speaking our definition of $d_A(\phi)$ depends on the choice of regular Haar
measure $\mu$. Note that {\tt mathlib} offers a fixed Borel regular Haar measure
{\tt MeasureTheory.Measure.haar} on any locally compact topological group
and the actual definition of $d_A$ in the code uses this definition.
Note also that the code defines everything for multiplicative groups
and uses {\tt @[to\_additive]} to deduce the corresponding results
for additive groups.

Here are some basic results about this construction. In all of them,
$A$ is a locally compact topological group and $\phi:A\to A$ is
a group isomorphism and a homeomorphism. The first lemma shows that
the definition of $d_A(\phi)$ is indeed independent of the choice of Haar measure.

\inputleannode{MeasureTheory.addEquivAddHaarChar_eq}


\inputleannode{MeasureTheory.addEquivAddHaarChar_map}


We can of course also pull a Haar measure $\mu$ back along a homeomorphism $\phi$,
giving a measure $\phi^*\mu$ such that $\phi^*\mu(X)=\mu(\phi(X))$.

\inputleannode{MeasureTheory.addEquivAddHaarChar_comap}


Now let $\mu$ be any regular additive Haar measure on $A$.

\inputleannode{MeasureTheory.addEquivAddHaarChar_smul_preimage}


\inputleannode{MeasureTheory.addEquivAddHaarChar_smul_integral_map}


We also have the following variant:

\inputleannode{MeasureTheory.addEquivAddHaarChar_smul_integral_comap}


Note that as a consequence of lemma~\ref{MeasureTheory.addEquivAddHaarChar_smul_preimage},
if $X$ is a Borel subset of $A$ with positive finite measure then we can read
off $d_A(\phi)$ by $d_A(\phi)=\mu(X)/\mu(\phi^{-1}(X))$, and hence also by
$d_A(\phi)=\mu(\phi(X))/\mu(X)$. A nice special case is when
$\mu(X)=1$, in which case we have $d_A(\phi)=\mu(\phi(X))$ for all $\phi$,
or $d_A(\phi)=1/\mu(\phi^{-1}(X))$.
Similarly, by lemma~\ref{MeasureTheory.addEquivAddHaarChar_smul_integral_map},
if $f$ is a measurable function with $0<\int f(x)d\mu(x)<\infty$ then
we can read off $d_A(\phi)$ by $d_A(\phi)=(\int f(x)d\mu(x))/(\int f(x)\phi_*\mu(x))$.
Note that {\tt mathlib} supplies such $f$ with the function {\tt exists\_continuous\_nonneg\_pos}.
The following are also straightforward.

\inputleannode{MeasureTheory.addEquivAddHaarChar_refl}


\inputleannode{MeasureTheory.addEquivAddHaarChar_trans}


\subsection{Scaling Haar measure on a ring}

Now let $R$ be a locally compact topological ring. The \emph{Haar character} of $R$,
or more precisely the \emph{left Haar character} of $R$, is a group homomorphism
$R^\times\to\R^\times$ defined in the following way. If $u\in R^\times$ then left multiplication
by $u$, namely the map $\ell_u:(R,+)\to(R,+)$ defined by $\ell_u(r)=ur$, is a homeomorphism and
an additive automorphism of $(R,+)$, so the preceding theory applies to $\ell_u$.

\inputleannode{MeasureTheory.ringHaarChar}

Lemmas~\ref{MeasureTheory.addEquivAddHaarChar_refl}
and~\ref{MeasureTheory.addEquivAddHaarChar_trans} immediately imply that $\delta_R$ is a group
homomorphism from $R^\times$ to $\R_{>0}$.
Also immediate from previous lemmas is

\inputleannode{MeasureTheory.ringHaarChar_mul_integral}


\inputleannode{MeasureTheory.ringHaarChar_mul_volume}


The next result lies a little deeper.

\inputleannode{MeasureTheory.ringHaarChar_continuous}


\section{Examples}

We discuss some examples of Haar characters.

\inputleannode{MeasureTheory.ringHaarChar_real}


\inputleannode{MeasureTheory.ringHaarChar_complex}


\inputleannode{MeasureTheory.ringHaarChar_padic}


\begin{remark}
If $R$ is a finite extension of $\Q_p$ then $\delta_R(u)$
is the norm on $R$ normalised in the following way:
$\delta_R(\varpi)=q^{-1}$, where $\varpi$ is a uniformiser
and $q$ is the size of the (finite) residue field. In fact the same is true
for any nonarchimedean local field. The proof is
the same as for $\Q_p$. Right now this is difficult to state in Lean because
there is still some discussion about the definition of a nonarchimedean local field.
\end{remark}

\section{Algebras}

  Say $F$ is a locally compact topological field (for example $\R$ or $\bbC$ or $\Q_p$), $V$
  is a finite-dimensional $F$-vector space, and $\phi:V\to V$ is an invertible $F$-linear map.
  Then $V$ with its module topology (which is the product topology if one picks a basis)
  is a locally compact topological abelian group, and $\phi$ is additive.
  One can check that linearity implies continuity (this is {\tt IsModuleTopology.continuous\_of\_linearMap} in mathlib),
  so in fact $\phi$ is a homeomorphism
  and our theory applies. The following lemma gives a formula for the scale factor $d_V(\phi)$.

\inputleannode{MeasureTheory.addEquivAddHaarChar_eq_ringHaarChar_det}


Now say $F$ is still a locally compact topological field, and that $R$ is a (possibly
non-commutative) $F$-algebra. Recall that this means that ($R=0$ or) $F$ lies in the centre of $R$.
Assume that $R$ is finite-dimensional as an $F$-vector space. Then if we give $R$ the
$F$-module topology (which is just the product topology if we pick a basis) then it is known
that $R$ becomes a topological ring. Now say $u\in R^\times$, and
recall $\ell_u:R\to R$ is left multiplication by $u$. Then $\ell_u$ is easily checked to be
an $F$-linear homeomorphism.

  \inputleannode{MeasureTheory.algebra_ringHaarChar_eq_ringHaarChar_det}


\section{Left and right multiplication}

If $R$ is a locally compact topological ring, and if multiplication on $R$ is not commutative,
then left and right multiplication by an element of~$R$ can scale Haar measure in different ways.
For example if $R$ is the upper-triangular $2\times 2$ matrices with real
entries, then left multiplication by $\begin{pmatrix}a&0\\0&1\end{pmatrix}$
sends $\begin{pmatrix}x&y\\0&z\end{pmatrix}$ to $\begin{pmatrix}ax&ay\\0&z\end{pmatrix}$
and thus scales $R$'s additive
Haar measure by $|a|^2$, but right multiplication by $\begin{pmatrix}a&0\\0&1\end{pmatrix}$
sends $\begin{pmatrix}x&y\\0&z\end{pmatrix}$ to $\begin{pmatrix}ax&y\\0&z\end{pmatrix}$
and thus scales $R$'s additive Haar measure by a factor of $|a|$.

What's going on here is that if we regard left and right multiplication as $\R$-linear
maps from $R$ to $R$, then their associated matrices with respect to the obvious basis
are $diag(a,a,1)$ and $diag(a,1,1)$, which have different determinants.

However, if $k$ is now any field and if $B$ is a finite-dimensional central
simple algebra over $k$ (for example a quaternion algebra, the case we'll care about later),
and if $u\in B^\times$ then $x\mapsto ux$ and $x\mapsto xu$
are both $k$-linear endomorphisms of $B$, and I claim that they have
the same determinant.

\inputleannode{IsSimpleRing.mulLeft_det_eq_mulRight_det}


\inputleannode{IsSimpleRing.ringHaarChar_eq_addEquivAddHaarChar_mulRight}


\section{Finite Products}

Here are two facts which we will need about products.

\inputleannode{MeasureTheory.addEquivAddHaarChar_prodCongr}


\inputleannode{MeasureTheory.addEquivAddHaarChar_piCongrRight}


\inputleannode{MeasureTheory.ringHaarChar_prod}


\inputleannode{MeasureTheory.ringHaarChar_pi}


\section{Some measure-theoretic preliminaries}

\inputleannode{Topology.IsOpenEmbedding.isHaarMeasure_comap}


\inputleannode{Topology.IsOpenEmbedding.regular_comap}


\inputleannode{MeasureTheory.mulEquivHaarChar_eq_one_of_compactSpace}


\inputleannode{MeasureTheory.addEquivAddHaarChar_eq_addEquivAddHaarChar_of_isOpenEmbedding}


\section{Restricted products}

Now say $A=\prod'_i A_i$ is the restricted product of a collection of types $A_i$
  with respect to the subsets $C_i$. Recall that this is the subset of $\prod_i A_i$
  consisting of y Say $B=\prod'_i B_i$ is the restricted
  product of types $B_i$ over the same index set, with respect to
  subsets $D_i$. Say $\phi_i:A_i\to B_i$ are functions
  with the property that $\phi_i(C_i)\subseteq D_i$ for all but finitely many $i$.
It is easily checked that the $\phi_i$ induce a function $\phi:=\prod'_i\phi_i:A\to B$. It is also easily
checked that if all the $A_i$ and $B_i$ are groups or rings or $R$-modules, the $C_i$ and $D_i$
are subgroups or subrings or submodules,
and the $\phi_i$ are group or ring or module homomorphisms, then $\phi$ is a group or ring or
module homomorphism. However topological facts lie a little deeper.

\inputleannode{Continuous.restrictedProduct_congrRight}


We now focus on the case that $B_i=A_i$ are locally compact groups, $D_i=C_i$ are compact
open subgroups, and $\phi_i:A_i\to A_i$ are group isomorphisms and homeomorphisms sending
$C_i$ onto $C_i$ for all but finitely many $i$. Then the restricted product $A:=\prod'A_i$
of the $A_i$ with respect to the $C_i$ is also a locally compact topological group, and the
restricted product $\phi=\prod'\phi_i$ of the $\phi_i$ is a group isomorphism and homeomorphism,
so we can ask how $d_A(\phi)$ compares to the $d_{A_i}(\phi_i)$.

First note that $d_{A_i}(\phi_i)=1$ for all the $i$ such that $\phi_i(C_i)=C_i$, as
$d_{A_i}(\phi_i)$ can be computed as $\mu(\phi_i(C_i))/\mu(C_i)$ and $\mu(C_i)$ is
guaranteed to have positive finite measure as it is open and compact. Hence the product
$\prod_id_{A_i}\phi_i$ is a finite product, in the sense that all but finitely many terms are 1.
The following theorem shows that the value of this product is $d(\phi)$.

\inputleannode{MeasureTheory.addEquivAddHaarChar_restrictedProductCongrRight}


As a special case, if $R$ is the restricted product of a collection of topological rings $R_i$
  (not necessarily commutative) each equipped with a compact open subring $C_i$, then
  we have

\inputleannode{MeasureTheory.ringHaarChar_restrictedProduct}


\section{Adeles}

We finish this miniproject by proving some results about Haar characters for
algebras over adele rings.
So let $K$ be a number field and let $\A_K$ be the adeles of $K$.
We will prove some theorems about $\A_K$-algebras $R$ which are finite
and free as $\A_K$-modules. Such algebras can be given the $\A_K$-module topology
and this makes them into locally compact topological rings. In fact we shall only be concerned
in applications with algebras of the form $B\otimes_K\A_K$ where $B$ is a finite-dimensional
$K$-algebra. So fix such a $B$, and write $B_{\A}$ for $B\otimes_K\A_K$. Let us first
deal with a subtlety. Recall that if $K\subseteq L$ are number fields, then $\A_L$ is a
module-finite $\A_K$-algebra and hence an $\A_K$-module, and
theorem~\ref{NumberField.AdeleRing.baseChange_moduleTopology}
tells us that $\A_L$ has the $\A_K$-module topology. Thus the next lemma applies.

\inputleannode{IsModuleTopology.continuous_bilinear_of_finite_left}


\inputleannode{NumberField.AdeleRing.ModuleBaseChangeContinuousAddEquiv}


As a consequence, if $B$ is a $K$-algebra then we can think of $B_{\A}$ as either $B\otimes_K\A_K$
with the $\A_K$-module topology or as $B\otimes_{\Q}\A_{\Q}$ with the $\A_{\Q}$-module
topology. Note that this isomorphism commutes with the inclusions from $B$ into these rings.
But Lean is picky about these things so we'll have to be careful.

\inputleannode{NumberField.AdeleRing.isCentralSimple_addHaarScalarFactor_left_mul_eq_right_mul}


The previous theorem only applies to inner forms of matrix algebras, but the below theorem,
a generalization of the adelic product formula, is valid for any finite-dimensional
$K$-algebra. Before we state it let's remind ourselves of the product formula for $\Q$,
and restate it in the language of these Haar characters.

\inputleannode{MeasureTheory.ringHaarChar_adeles_rat}


Now $\A_{\Q}$ is nonzero a $\Q$-algebra and hence we have an inclusion $\Q^\times\to\A_{\Q}^\times$.
Here is our reinterpretation of the product formula.

\inputleannode{MeasureTheory.ringHaarChar_adeles_units_rat_eq_one}


  Next we generalize this to finite-dimensional $\Q$-vector spaces.

  So say $V$ is an $N$-dimensional $\Q$-vector space,
  and define $V_{\A}:= V\otimes_{\Q}\A_{\Q}$ with its $\A_{\Q}$-module topology.
  If we choose an isomorphism $V\cong\Q^N$ then $V_{\A}\cong\A_{\Q}^N$
  as an additive topological abelian group. In particular, $V_{\A}$ is locally compact.

  Fix a $\Q$-linear automorphism $\phi:V\to V$. By base extension $\phi$ induces
  an $\A_{\Q}$-linear automorphism $\phi_{\A}$ of $V_{\A}$ which is also a homeomorphism of $V_{\A}$
  if $V_{\A}$ is given the module topology as an $\A_{\Q}$-module. Our goal is

  \inputleannode{MeasureTheory.addHaarScalarFactor_tensor_adeles_eq_one}
  

  \inputleannode{NumberField.AdeleRing.units_mem_ringHaarCharacter_ker}
  

  \inputleannode{NumberField.AdeleRing.addEquivAddHaarChar_mulRight_unit_eq_one}
  
